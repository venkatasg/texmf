% Sample document showcasing practical-presentation.cls features
% Compile with: lualatex sample-presentation.tex

\documentclass[biblatex, light]{practical-presentation}
% Options demonstrated:
% - biblatex: enables bibliography support
% - light: light theme (use 'dark' for dark theme with thinner fonts)
% Try changing 'light' to 'dark' to see the alternative theme!

\title{Sample Presentation}
\subtitle{Demonstrating the Practical-Presentation Class}
\author{Jane Researcher}
\institute{Department of Example Studies\\University of Samples}
\date{\today}

% Bibliography
\addbibresource{references.bib}

\begin{document}

% ================ Title Frame ================
\begin{frame}
\maketitle
\end{frame}

% ================ Table of Contents ================
\begin{frame}{Outline}
\tableofcontents
\end{frame}

% ================ Section 1 ================
\section{Typography Features}

\begin{frame}{Font Families and Variants}

The presentation uses \textbf{Concourse 3} (light theme) or \textbf{Concourse 2} (dark theme) as the main sans serif font with several variants:

\begin{itemize}
\item {\sansc Small Caps} using \texttt{\sansc}
\item {\sancs All Caps} using \texttt{\sancs}
\item {\rmfamily Serif text using Equity A} with \texttt{\rmfamily}
\item {\rmsc Serif Small Caps} using \texttt{\rmsc}
\item \texttt{Monospace using Triplicate A Code}
\end{itemize}

Mathematical expressions use FiraMath: $E = mc^2$ and $\int_{0}^{\infty} e^{-x} \, dx = 1$

\end{frame}

\begin{frame}{Special Numbering Fonts}

The class provides fancy circle numbering fonts from Concourse T3 Index:

\begin{itemize}
\item White circles: {\nfwc 1234567890}
\item Black circles: {\nfbc 1234567890}
\end{itemize}

These appear in:
\begin{enumerate}
\item Enumerated lists (like this!)
\item Table of contents numbering
\item Any place you use the \texttt{\nfwc} or \texttt{\nfbc} commands
\end{enumerate}

\end{frame}

% ================ Section 2 ================
\section{Color and Themes}

\begin{frame}{Theme Options}

\textbf{Light Theme} (current):
\begin{itemize}
\item Dark grey text on pale grey background
\item Concourse 3 font (regular weight)
\item Best for well-lit rooms
\end{itemize}

\textbf{Dark Theme} (\texttt{dark} option):
\begin{itemize}
\item Pale grey text on dark grey background
\item Concourse 2 font (thinner weight for better contrast)
\item Best for darkened presentation rooms
\end{itemize}

\alert{Change the document class option to switch themes!}

\end{frame}

\begin{frame}{Custom Colors}

The class defines several color palettes:

\textbf{Accent Colors:}
\begin{itemize}
\item \alert{Alert text in MokaRed}
\item \href{https://example.com}{Hyperlinks in SkyBlue}
\item \textcolor{burntorange}{Burnt Orange accent}
\item \textcolor{ithacablue}{Ithaca Blue accent}
\end{itemize}

\textbf{Tol Color Scheme:}
\textcolor{TolIndigo}{Indigo}, \textcolor{TolCyan}{Cyan}, \textcolor{TolTeal}{Teal}, \textcolor{TolGreen}{Green}, \textcolor{TolOlive}{Olive}, \textcolor{TolSand}{Sand}, \textcolor{TolRose}{Rose}, \textcolor{TolWine}{Wine}, \textcolor{TolPurple}{Purple}

\end{frame}

% ================ Section 3 ================
\section{Beamer Features}

\begin{frame}{Blocks and Environments}

\begin{block}{Standard Block}
This is a standard block environment with the moloch theme styling and fill option.
\end{block}

\begin{alertblock}{Alert Block}
This is an alert block using the MokaRed accent color.
\end{alertblock}

\begin{exampleblock}{Example Block}
This is an example block for demonstrations and examples.
\end{exampleblock}

\end{frame}

\begin{frame}{Lists and Spacing}

\textbf{Itemized Lists:}
\begin{itemize}
\item First item with custom spacing
\item Second item using parskip package
\item Third item with ragged-right alignment
\end{itemize}

\textbf{Enumerated Lists:}
\begin{enumerate}
\item Lists use fancy circle numbering
\item From the Concourse T3 Index font
\item Providing visual distinction
\end{enumerate}

\end{frame}

\begin{frame}{Progress Bars}

Notice the progress bar features:
\begin{itemize}
\item Frame title progress bar (top of each frame)
\item Section page progress bars
\item Thin 1pt line width for subtle effect
\item MokaRed accent color
\end{itemize}

These are configured in the moloch theme setup with custom line widths.

\end{frame}

% ================ Section 4 ================
\section{Tables and Figures}

\begin{frame}{Table Example}

Tables automatically use monospaced uppercase numbers:

\begin{table}
\centering
\begin{tabular}{lrrr}
\toprule
Category & Value 1 & Value 2 & Total \\
\midrule
Alpha & 123 & 456 & 579 \\
Beta & 789 & 234 & 1023 \\
Gamma & 345 & 678 & 1023 \\
\bottomrule
\end{tabular}
\end{table}

The \texttt{booktabs} package provides professional table formatting.

\end{frame}

\begin{frame}{Figure Example}

Figures use custom caption formatting (empty label by default):

\begin{figure}
\centering
\rule{0.6\textwidth}{0.3\textwidth}
\caption{Placeholder demonstrating figure layout}
\end{figure}

Graphics path is preset to \texttt{figures/} directory.

\end{frame}

\begin{frame}{PGFPlots Integration}

The class loads \texttt{pgfplots} for data visualization:

\begin{center}
\begin{tikzpicture}
\begin{axis}[
    width=0.7\textwidth,
    height=0.4\textwidth,
    xlabel={X axis},
    ylabel={Y axis},
    grid=major,
]
\addplot[TolIndigo, thick, mark=*] coordinates {
    (0,0) (1,1) (2,1.5) (3,2) (4,2.3) (5,2.5)
};
\addplot[MokaRed, thick, mark=square] coordinates {
    (0,0) (1,0.5) (2,1.2) (3,2.1) (4,3.2) (5,4.5)
};
\end{axis}
\end{tikzpicture}
\end{center}

\end{frame}

% ================ Section 5 ================
\section{Linguistic Examples}

\begin{frame}{Linguex Support}

The \texttt{linguex} package is loaded with custom formatting:

\ex. This is a simple linguistic example.

\ex.
\a. Multiple subexamples
\b. With proper spacing
\c.\label{ex:2c} And fancy numbering

Examples can be labelled and referred back: \ref{ex:2c}

\end{frame}

% ================ Section 6 ================
\section{Footnotes and Citations}

\begin{frame}[fragile]{Footnote Commands}

Regular footnotes work as expected\footnote{This is a regular footnote with superscript numbering.} with custom formatting.

You can also use blind footnotes without numbers:\blfootnote{This is a blind footnote created with \texttt{\string\blfootnote}.}

When \texttt{biblatex} is enabled, use \texttt{\string\blfootcite} for citation footnotes without markers.

\end{frame}

\begin{frame}[fragile]{Bibliography Support}

When the \texttt{biblatex} option is enabled:

\begin{itemize}
\item Author-year citation style.
\item Maximum 1 cite name, truncated with "et al."
\item Custom bibliography formatting
\item Footnote-sized citations
\item Normal-sized bibliography entries
\item URLs, DOIs, and eprints disabled
\end{itemize}

Use \texttt{\string\cite}, \texttt{\string\textcite}, \texttt{\string\parencite}, etc. for citations.

Example citations: \citep{Smith:01} and \citet{Mueller:02}.

\end{frame}

% ================ Section 7 ================
\section{Special Frames}

\begin{frame}[standout]
Standout Frame

Use \texttt{[standout]} option for emphasis slides with custom background and large upright text.
\end{frame}

\begin{frame}[plain]
\centering
\vspace{2cm}
{\Huge Plain Frame}

\vspace{1cm}
Use \texttt{[plain]} option for frames without header, footer, or frame title.

Useful for full-screen images or special layouts.
\end{frame}

% ================ Section 8 ================
\section{Advanced Features}

\begin{frame}{Moloch Theme Configuration}

The class uses the moloch beamer theme with:

\begin{itemize}
\item Progress bar in frame titles
\item Section pages with progress bars
\item All small caps title formatting
\item Regular formatting for title page titles
\item Block fill styling
\item Custom MokaRed progress color
\item 1pt line width for all progress elements
\end{itemize}

\end{frame}

\begin{frame}[fragile]{Appendix Support}

The \texttt{appendixnumberbeamer} package is loaded, so you can add backup slides after:

\begin{verbatim}
\appendix
\section{Backup Slides}
\end{verbatim}

Appendix frames won't affect the frame counter shown to the audience.

\end{frame}

% ================ Summary ================
\section{Summary}

\begin{frame}[standout]

This sample document demonstrates all the unique features of the \texttt{practical-presentation} class.

Thank You!

Questions?
\end{frame}

% =============== References ===============

{
\molochset{numbering=none}
\section*{references}
\begin{frame}[noframenumbering]{References}
\printbibliography[heading=none]
\end{frame}
}

% ================ Appendix ================
\appendix

{
\molochset{progressbar=none}
\section{Backup Slides}

\begin{frame}{Additional Information}

This is a backup slide in the appendix. It won't affect the total frame count displayed in the presentation.

\textbf{Additional Resources:}
\begin{itemize}
\item Moloch beamer theme documentation
\item Practical Typography by Matthew Butterick
\item Paul Tol's color schemes
\item Linguex package documentation
\end{itemize}

\end{frame}
}

\end{document}
